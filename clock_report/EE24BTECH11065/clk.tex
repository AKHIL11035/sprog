\documentclass{article}
\usepackage{graphicx}
\usepackage{amsmath}
\usepackage{hyperref}
\usepackage{xcolor}
\usepackage{listings}

\title{\textbf{Digital Clock}}
\author{Spoorthi Yellamanchali - EE24BTECH11065}
\date{\today}

\begin{document}

\maketitle

% Define C code style with colors
\lstdefinestyle{CStyle}{
    language=C,
    basicstyle=\ttfamily\footnotesize,
    keywordstyle=\color{blue},
    stringstyle=\color{red},
    commentstyle=\color{green!50!black},
    numbers=left,
    numberstyle=\tiny\color{gray},
    stepnumber=1,
    breaklines=true,
    frame=single,
    backgroundcolor=\color{gray!10}
}


\section{Aim}
 To design and implement  a digital clock using an Arduino Uno.

\section{Components Used}
Table \ref{tab:components} lists the components used in this project along with their purpose and connections.

\begin{tabular}[12pt]{ |c| c| c |}
    \hline
    \textbf{Variable} & \textbf{Description} & \textbf{values}\\ 
    \hline
    $\vec{V}$ & Quadratic form of the matrix & $\myvec{1 & 0 \\ 0 & 1} $\\
    \hline
    $\vec{u}$ & Linear coefficient vector & $\myvec{0 \\ 0} $\\
    \hline
    f & constant term & -4 \\ 
    \hline
    $\vec{m}$ & The direction vector of line & $\myvec{1 \\ 0}$\\
    \hline
     $\vec{h}$ & Point on line & \myvec{2 \\ 0} \\
     \hline
\end{tabular}


\section{Working Principle}
\begin{itemize}
    \item This clock operates using a multiplexing technique, where only one 7-segment display is activated at a time.
    \item The Arduino rapidly switches between displays, creating an illusion of a continuous display to the human eye. The 7447 decoder is used to convert BCD values into 7-segment outputs.
\end{itemize}
 

\section{Multiplexing Explanation}
Multiplexing is used to drive multiple 7-segment displays with fewer microcontroller pins. By cycling through displays quickly, we reduce the number of required I/O pins. This technique involves:
\begin{itemize}
    \item Assigning a unique enable pin to each display.
    \item Sending the appropriate BCD value to the 7447 decoder.
    \item Activating one display at a time while others are off.
\end{itemize}
This process is performed fast enough that the human eye perceives a steady display.
\section{Connections}
\subsection{BCD Pins (Arduino to 7447 Decoder)}
The following table describes the connection between the Arduino and the 7447 BCD decoder:

\begin{tabular}{|l|c l|c l|}
    \hline
    & (a) & & (b) & \\
    \hline
    Postulate 2 & $A + 0 = A$ & & $A \cdot 1 = A$ & \\
    \hline
    Postulate 5 & $A + A' = 1$ & & $A
    \cdot A' = 0$ & \\
    \hline
    Theorem 1 & $A + A = A$ & & $A \cdot A = A$ & \\
    \hline
    Theorem 2 & $A + 1 = 1$ & & $A \cdot 0 = 0$ & \\
    \hline
    Theorem 3, involution & $(A')' = A$ & & - & \\
    \hline
    Postulate 3, commutative & $A + B = B + A$ & & $AB = BA$ & \\
    \hline
    Theorem 4, associative & $A + (B + C) = (A + B) + C$ & & $A(BC) = (AB)C$ & \\
    \hline
    Postulate 4, distributive & $A(B + C) = AB + AC$ & & $A + BC = (A + B)(A + C)$ & \\
    \hline
    Theorem 5, DeMorgan & $(A + B)' = A' B'$ & & $(AB)' = A' + B'$ & \\
    \hline
    Theorem 6, absorption & $A + AB = A$ & & $A(A + B) = A$ & \\
    \hline
\end{tabular}


\subsection{7-Segment Display Selection}
Each of the six 7-segment displays is controlled through multiplexing:

\begin{table}[h]
    \centering
    \begin{tabular}{|c|c|c|c|}
        \hline
        \textbf{Display} & \textbf{Arduino Pin} & \textbf{ATmega328P Pin} & \textbf{Function} \\
        \hline
        DISP1 & PD6 & 10 & Selects Hour Tens display \\
        DISP2 & PD7 & 11 & Selects Hour Units display \\
        DISP3 & PB0 & 8  & Selects Minute Tens display \\
        DISP4 & PB1 & 9  & Selects Minute Units display \\
        DISP5 & PB2 & 10 & Selects Second Tens display \\
        DISP6 & PB3 & 11 & Selects Second Units display \\
        \hline
    \end{tabular}
    \caption{7-Segment Display Selection}
    \label{tab:display_selection}
\end{table}


The microcontroller rapidly switches between these displays using multiplexing, ensuring that all digits appear visible simultaneously.
\subsection{Push Button Connections}
The push buttons allow manual time adjustment:

\begin{table}[h]
    \centering
    \begin{tabular}{|c|c|c|c|}
        \hline
        \textbf{Button} & \textbf{Arduino Pin} & \textbf{ATmega328P Pin} & \textbf{Function} \\
        \hline
        HOUR\_BUTTON   & PB4 & 12 & Increments the hour \\
        MIN\_BUTTON    & PC0 & A0 & Increments the minute \\
        SEC\_BUTTON    & PC1 & A1 & Increments the second \\
        \hline
    \end{tabular}
    \caption{Push Button Connections}
    \label{tab:push_buttons}
\end{table}

Each button uses an internal pull-up resistor, meaning it reads HIGH when unpressed and LOW when pressed. A software debounce mechanism prevents false triggers.







\section{Code implementation}
This document explains the modular structure of the **Arduino-based digital clock** that uses a **7447 BCD to 7-segment decoder** for display multiplexing.

\subsection{Modules of the Code}

\subsubsection{BCD Encoder Module}
The function \texttt{setBCD(uint8\_t num)} converts a decimal number into a 4-bit BCD format and sends it to the 7447 decoder.

\begin{lstlisting}[style=CStyle]
void setBCD(uint8_t num) {
    // Clear previous BCD data while preserving unaffected bits
    PORTD = (PORTD & 0xC3) | ((num & 0x0F) << 2);  
}
\end{lstlisting}

\subsubsection{Display Selection Module}
The function \texttt{selectDisplay(uint8\_t disp)} selects one of the six 7-segment displays.

\begin{lstlisting}[style=CStyle]
void selectDisplay(uint8_t disp) {
    // Turn off all display selection lines before enabling the required one
    PORTD &= ~((1 << DISP1) | (1 << DISP2));
    PORTB &= ~((1 << DISP3) | (1 << DISP4) | (1 << DISP5) | (1 << DISP6));

    // Activate only the required display based on the index
    switch (disp) {
        case 0: PORTD |= (1 << DISP1); break;  // Hour Tens
        case 1: PORTD |= (1 << DISP2); break;  // Hour Units
        case 2: PORTB |= (1 << DISP3); break;  // Minute Tens
        case 3: PORTB |= (1 << DISP4); break;  // Minute Units
        case 4: PORTB |= (1 << DISP5); break;  // Second Tens
        case 5: PORTB |= (1 << DISP6); break;  // Second Units
    }
}
\end{lstlisting}

\subsubsection{Multiplexed Display Module}
The function \texttt{displayTime()} cycles through all displays and updates them.

\begin{lstlisting}[style=CStyle]
void displayTime() {
    // Extract individual digits for hours, minutes, and seconds
    uint8_t digits[6] = {
        hours / 10, hours % 10,  // Hour Tens and Units
        minutes / 10, minutes % 10,  // Minute Tens and Units
        seconds / 10, seconds % 10   // Second Tens and Units
    };

    static uint8_t currentDisplay = 0;  // Keeps track of the active display
    
    setBCD(digits[currentDisplay]);  // Send the digit value to the 7447 decoder
    selectDisplay(currentDisplay);   // Activate the respective 7-segment display
    currentDisplay = (currentDisplay + 1) % 6;  // Cycle through displays
}
\end{lstlisting}

\subsubsection{Button Handling Module}
This function reads the button inputs and updates the time accordingly.

\begin{lstlisting}[style=CStyle]
void checkButtons() {
    // Check if the hour button is pressed
    if (!(PINB & (1 << HOUR_BUTTON))) {  // Active low button press
        _delay_ms(50);  // Debounce delay
        if (!(PINB & (1 << HOUR_BUTTON))) {  // Ensure the button is still pressed
            hours = (hours + 1) % 24;  // Increment the hour, rolling over at 24
        }
    }

    // Check if the minute button is pressed
    if (!(PINC & (1 << MIN_BUTTON))) {
        _delay_ms(50);
        if (!(PINC & (1 << MIN_BUTTON))) {
            minutes = (minutes + 1) % 60;  // Increment minutes, rolling over at 60
        }
    }
    
    // Check if the second button is pressed
    if (!(PINC & (1 << SEC_BUTTON))) {
        _delay_ms(50);
        if (!(PINC & (1 << SEC_BUTTON))) {
            seconds = (seconds + 1) % 60;  // Increment seconds, rolling over at 60
        }
    }
}
\end{lstlisting}
\subsection{Compilation and Upload Process}
The code for the digital clock was written in **Embedded C** and compiled using the **AVR-GCC** toolchain. The following steps were followed:

\begin{enumerate}
    \item The code was written and compiled using the AVR-GCC `make` command to generate the binary file.
    \item The compiled binary was then transferred into the **precompiled section** of **ArduinoDroid**, an Android-based IDE for Arduino development.
    \item Using ArduinoDroid, the binary was uploaded to the **Arduino Uno** microcontroller, ensuring the clock's execution.
\end{enumerate}

This method allows seamless integration of **AVR low-level programming** with **Arduino-based deployment** for better flexibility and control.

\section{Conclusion}
This project successfully implements a digital clock using an Arduino Uno and a 7447 BCD decoder. Multiplexing effectively reduces hardware requirements while maintaining clear and readable digit output. The modular code structure ensures easy understanding and modifications.

\end{document}

