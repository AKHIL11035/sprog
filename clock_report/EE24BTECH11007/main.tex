\documentclass{article}
\usepackage{graphicx}
\usepackage{amsmath}
\usepackage{listings}
\usepackage{hyperref}

\title{Digital Clock}
\author{ArnavYadnopavit\\EE24BTECH11007}
\date{\today}

\begin{document}

\maketitle

\section{Introduction}
This report describes the implementation of a digital clock using an Arduino Uno board with a 7447 BCD decoder. The project uses a multiplexed seven-segment display to show the current time, which can be adjusted using pushbuttons.

\section{Hardware Connections}
The following table lists the hardware connections between the microcontroller and other components:

\begin{table}[h]
    \centering
    \begin{tabular}{|c|c|c|}
        \hline
        \textbf{Microcontroller Pin} & \textbf{Connected Component} & \textbf{Function} \\
        \hline
        D2-D5 & 7447 BCD Decoder & BCD Digit Selection \\
        D6, D7 & Digit Enable & Selects Active Display Digit \\
        PB0-PB3 & Digit Enable & Enables Different Digits \\
        PB4 & Hour Button & Increments Hours \\
        PB5 & Minute Button & Increments Minutes \\
        \hline
    \end{tabular}
    \caption{Pin Connections}
    \label{tab:connections}
\end{table}

\section{Software Implementation}

\subsection{Timer Configuration}
The clock utilizes Timer0 in CTC (Clear Timer on Compare Match) mode to generate a 1ms time base. This is essential for keeping track of time and handling display multiplexing.

\begin{lstlisting}[language=C, caption=Timer0 Configuration]
TCCR0A = (1 << WGM01);  // CTC Mode
TCCR0B = (1 << CS01) | (1 << CS00);  // Prescaler 64
OCR0A = 249;  // 1ms interval
TIMSK0 |= (1 << OCIE0A);  // Enable Timer Interrupt
\end{lstlisting}

\subsection{Multiplexed Display}
Multiplexing is used to drive all six digits using fewer GPIO pins. Instead of having separate lines for each seven-segment display, only four BCD output lines (PD2-PD5) are used. Each digit is enabled sequentially using control signals (PD6, PD7, PB0-PB3). This rapid switching creates the illusion that all digits are continuously lit.

The multiplexing process involves:
\begin{itemize}
\item Setting the correct BCD value for a digit.
\item Enabling only one digit at a time.
\item Introducing a small delay before switching to the next digit.
\item Repeating the cycle continuously to maintain a steady display.
\end{itemize}

\begin{lstlisting}[language=C, caption=Digit Multiplexing]
void displayDigit(uint8_t digit, uint8_t position) {
    // Set the BCD output bits on PD2-PD5.
    PORTD = (PORTD & 0xC3) | ((digit & 0x0F) << 2);
    
    // Enable the appropriate digit.
    PORTD &= ~((1 << PD6) | (1 << PD7));
    PORTB &= ~((1 << PB0) | (1 << PB1) | (1 << PB2) | (1 << PB3));
    
    switch(position) {
        case 0: PORTD |= (1 << PD6); break;
        case 1: PORTD |= (1 << PD7); break;
        case 2: PORTB |= (1 << PB0); break;
        case 3: PORTB |= (1 << PB1); break;
        case 4: PORTB |= (1 << PB2); break;
        case 5: PORTB |= (1 << PB3); break;
    }
    
    _delay_ms(3);  // Short delay to reduce flicker.
}
\end{lstlisting}

\subsection{Time Keeping}
The system maintains time using a software counter that increments every second. The \texttt{millis()} function retrieves the elapsed time based on Timer0.

\begin{lstlisting}[language=C, caption=Time Keeping]
uint32_t millis(void) {
    uint32_t ms;
    cli();
    ms = timer_millis;
    sei();
    return ms;
}
\end{lstlisting}

\subsection{Button Handling with Debouncing}
Push buttons are connected to PB4 and PB5 for adjusting hours and minutes. Since mechanical switches cause bouncing effects, a software debounce mechanism is implemented using a time delay.

\begin{lstlisting}[language=C, caption=Button Debouncing]
void checkButtons(void) {
    bool hourState = (PINB & (1 << PB4)) != 0;
    bool minuteState = (PINB & (1 << PB5)) != 0;
    uint32_t currentMillis = millis();
    
    if (!hourState && lastHourState && ((currentMillis - lastHourPress) > debounceDelay)) {
        hours = (hours + 1) % 24;
        lastHourPress = currentMillis;
    }
    
    if (!minuteState && lastMinuteState && ((currentMillis - lastMinutePress) > debounceDelay)) {
        minutes = (minutes + 1) % 60;
        lastMinutePress = currentMillis;
    }
    
    lastHourState = hourState;
    lastMinuteState = minuteState;
}
\end{lstlisting}
For codes refer to:\\
\url{https://github.com/ArnavYadnopavit/EE1003/tree/main/clock}\\

\centering
Thank you
\end{document}


