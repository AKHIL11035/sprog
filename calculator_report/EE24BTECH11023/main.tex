\let\negmedspace\undefined
\let\negthickspace\undefined
\documentclass[journal]{IEEEtran}
\usepackage[a5paper, margin=10mm, onecolumn]{geometry}
%\usepackage{lmodern} % Ensure lmodern is loaded for pdflatex
\usepackage{tfrupee} % Include tfrupee package

\setlength{\headheight}{1cm} % Set the height of the header box
\setlength{\headsep}{0mm}     % Set the distance between the header box and the top of the text

\usepackage{gvv-book}
\usepackage{gvv}
\usepackage{cite}
\usepackage{amsmath,amssymb,amsfonts,amsthm}
\usepackage{algorithmic}
\usepackage{graphicx}
\usepackage{siunitx}
\usepackage{textcomp}
\usepackage{xcolor}
\usepackage{txfonts}
\usepackage{listings}
\usepackage{enumitem}
\usepackage{mathtools}
\usepackage{gensymb}
\usepackage{comment}
\usepackage[breaklinks=true]{hyperref}
\usepackage{tkz-euclide} 
\usepackage{listings}
% \usepackage{gvv}                                        
\def\inputGnumericTable{}                                 
\usepackage[latin1]{inputenc}                                
\usepackage{color}                                            
\usepackage{array}                                            
\usepackage{longtable}                                       
\usepackage{calc}                                             
\usepackage{multirow}                                         
\usepackage{hhline}                                           
\usepackage{ifthen}                                           
\usepackage{lscape}
\begin{document}

\bibliographystyle{IEEEtran}
\vspace{3cm}

\title{SCIENTIFIC CALCULATOR}
\author{EE24BTECH11023 - RASAGNA}

% \maketitle
% \newpage
% \bigskip
{\let\newpage\relax\maketitle}

\renewcommand{\thefigure}{\theenumi}
\renewcommand{\thetable}{\theenumi}
\setlength{\intextsep}{10pt} % Space between text and floats


\numberwithin{equation}{enumi}
\numberwithin{figure}{enumi}
\renewcommand{\thetable}{\theenumi}

\section{Objective}
\begin{itemize}
    \item The objective of this project is to design and implement a scientific calculator using an LCD and AVR-GCC on an Arduino microcontroller.
    \item The calculator will be capable of performing basic arithmetic operations, trigonometric functions, logarithmic calculations, and other scientific computations.
    \item The project aims to develop an efficient, user-friendly interface for input and output, utilizing an LCD for display and a keypad for user interaction.
    \item Through this project, we explore the integration of embedded systems with mathematical operations, optimizing performance using AVR-GCC for precise and efficient computation.
\end{itemize}


\section{Components and Equipment}

\begin{table}[h]
    \centering
    \renewcommand{\arraystretch}{1.2}
    \begin{tabular}{|c|l|c|}
        \hline
        \textbf{S.No} & \textbf{Component} & \textbf{Quantity} \\
        \hline
         1& Arduino  & 1 \\
         2& Breadboard & 2 \\
         3& LCD display & 1  \\
         4& USB A to USB B cable & 1 \\
         5& OTG adapter & 1 \\
         6& Jumper wires (Male-Male)& 50 \\
         7& Resistors \SI{15000}{\ohm} & 9 \\
         8& Push Buttons &25\\
        \hline
    \end{tabular}
\end{table}



\begin{itemize}
    \item \textbf{VSS} $\rightarrow$ GND (Ground)
    \item \textbf{VDD} $\rightarrow$ 5V (Power Supply)
    \item \textbf{V0} $\rightarrow$ 10k Potentiometer (middle pin) (Contrast Adjustment)
    \item \textbf{RS} $\rightarrow$ D7 (Register Select)
    \item \textbf{RW} $\rightarrow$ GND (Read/Write, set to Write)
    \item \textbf{E} $\rightarrow$ D6 (Enable)
    \item \textbf{D4} $\rightarrow$ D4 (Data Bit 4)
    \item \textbf{D5} $\rightarrow$ D5 (Data Bit 5)
    \item \textbf{D6} $\rightarrow$ D6 (Data Bit 6)
    \item \textbf{D7} $\rightarrow$ D7 (Data Bit 7)
    \item \textbf{A (LED+)} $\rightarrow$ 5V with a Resistor (Backlight Power)
    \item \textbf{K (LED-)} $\rightarrow$ GND (Backlight Ground)
\end{itemize}

\section{Working Principle-Software Implementation}
\subsection{Software implementation using ADC Values}
The scientific calculator utilizes the ADC (Analog-to-Digital Converter) of the Arduino microcontroller to read analog inputs, such as a voltage divider-based keypad. The ADC converts the analog input into a digital value, which is then processed to determine the corresponding keypress.\\
The ADC is configured in free-running mode with a prescaler to ensure accurate readings.

\begin{itemize}
    \item \textbf{ADC Initialization:} Configure the ADC in the AVR-GCC environment by setting the appropriate registers.
    \item \textbf{Reading ADC Values:} Continuously read the ADC values from the analog pin where the keypad is connected.
    \item \textbf{Mapping ADC Values to Keypad Inputs:} Compare the ADC readings against predefined threshold values to determine which key is pressed.
    \item \textbf{Processing Key Inputs:} Perform mathematical operations based on the detected keypress and update the LCD accordingly.
    \item \textbf{Displaying Results:} Output the computed result on the LCD using 4-bit mode communication.
\end{itemize}

\subsection{Debouncing}
\begin{enumerate}
    \item Mechanical buttons often generate noisy signals, causing bouncing, where the button state might change rapidly due to physical contact. This can cause multiple increments instead of just one.
    \item To prevent this, a debounce delay is added, which ensures that only one increment happens for each button press (or jumper wire tap).
    \item After detecting a state change, the program waits for a short period (e.g., 300 ms) before checking the button state again.
    \item This debounce delay helps ignore any unintended multiple presses from the same action.
\end{enumerate}


\section{Advantages of AVR-GCC}
\begin{itemize}
    \item Direct access to AVR registers which results in smaller and faster machine code.
    \item No C++ overhead (such as classes, virtual functions, and dynamic memory allocation).
    \item We can directly manipulate registers (e.g., PORTB, DDRC) for faster execution.
    \item No need for functions like digitalWrite(), which are slower than direct register access.
    \item No unnecessary code from high-level abstractions or unused libraries.
    \item Avoids unexpected behavior caused by library overhead.
    \item AVR-GCC is faster, more efficient, and gives complete hardware control. 
\end{itemize}

\section{Precautions}
\subsection{Hardware}
\begin{enumerate}
    \item Always connect a Resistor between Backlight pin of LCD and 5V.
    \item Connect the RW pin of the LCD to GND to keep it in write mode.
    \item Use proper grounding to prevent noise and ensure stable operation.
\end{enumerate}

\subsection{Software}
\begin{enumerate}
    \item Verify correct pin assignments in the code to match hardware connections.
    \item Use appropriate delays after sending commands to allow the LCD to process them.
    \item Ensure proper handling of buffer overflows when processing user input.
\end{enumerate}
\end{document}

