
\definecolor{codebg}{rgb}{0.95,0.95,0.95}
\lstset{
    backgroundcolor=\color{codebg},
    basicstyle=\ttfamily\footnotesize,
    keywordstyle=\color{blue}\bfseries,
    commentstyle=\color{green!50!black},
    numbers=left,
    numberstyle=\tiny\color{gray},
    frame=single,
    breaklines=true
}
\subsection{Library Inclusions}
\begin{lstlisting}[language=C]
#include <avr/io.h>
#include <util/delay.h>
#include <stdlib.h>
#include <stdbool.h>
#include <string.h>
#include <ctype.h>
#include <math.h>
#include "lcd.h"
\end{lstlisting}
\textbf{Explanation:}
\begin{itemize}
\item \texttt{avr/io.h}: Provides access to microcontroller registers.
\item \texttt{util/delay.h}: Used for generating delays.
\item \texttt{stdbool.h}: Provides Boolean logic support.
\item \texttt{math.h}: Standard math library functions.
\item \texttt{lcd.h}: Custom LCD driver for display operations.
\end{itemize}

\subsection{Mathematical Functions Implementation}
Custom iterative approximations are used for trigonometric and logarithmic functions.

\subsubsection{Sine Approximation (Taylor Series)}
\begin{lstlisting}[language=C]
float mySin(float x){
float rad = x * PI / 180.0;
float term = rad;
float sum = rad;
for (int n = 1; n < 10; n++){
term = -term * rad * rad / ((2 * n) * (2 * n + 1));
sum += term;
}
return sum;
}
\end{lstlisting}
\subsubsection{Cos function}
\begin{lstlisting}[language=C]
float myCos(float x){
float rad = x * PI / 180.0;
float term = 1;
float sum = 1;
for (int n = 1; n < 10; n++){
    term = -term * rad * rad / ((2 * n - 1) * (2 * n));
    sum += term;
  }
  return sum;
}
\end{lstlisting}
\subsubsection{Square Root Function using Newton method}
\begin{lstlisting}[language=C]
float mySqrt(float x){
  if (x < 0) return -1;
  float guess = x / 2.0;
  for (int i = 0; i < 10; i++){
    guess = (guess + x / guess) / 2.0;
  }
  return guess;
}
\end{lstlisting}
\subsubsection{Natural logarithm}
\begin{lstlisting}[language=C]
float mySqrt(float x){
  if (x < 0) return -1;
  float guess = x / 2.0;
  for (int i = 0; i < 10; i++){
    guess = (guess + x / guess) / 2.0;
  }
  return guess;
}

}
\end{lstlisting}
\subsubsection{Exponential Approximation (Euler's Method)}
\begin{lstlisting}[language=C]
float myExp(float x){
int N = 100;
float h = x / N;
float y = 1.0;
for (int i = 0; i < N; i++){
y = y * (1 + h);
}
return y;
}
\end{lstlisting}

\subsection{Expression Parsing (Recursive Descent)}
Expressions are evaluated using a recursive descent parser.
\begin{lstlisting}[language=C]
float parseExpression(const char* s, int *pos) {
skipSpaces(s, pos);
float value = parseTerm(s, pos);
skipSpaces(s, pos);
while (s[*pos] == '+' || s[*pos] == '-') {
char op = s[(*pos)++];
float term = parseTerm(s, pos);
if (op == '+') value += term;
else value -= term;
skipSpaces(s, pos);
}
return value;
}
\end{lstlisting}

Parses and evaluates arithmetic expressions recursively.

\subsection{Button Handling}
Each button press is detected and processed accordingly.
\begin{lstlisting}[language=C]
bool button_pressed(uint8_t pinMask) {
return !(BUTTON_PORT & pinMask);
}
\end{lstlisting}

Checks if a button is pressed based on its pin state.

\subsection{LCD Update}
The LCD is updated with the current input string.
\begin{lstlisting}[language=C]
void updateLCD(void) {
lcd_clear();
lcd_print(input);
}
\end{lstlisting}

Clears and refreshes the display with new input.

\subsection{Main Execution Loop}
The program runs continuously, reading button inputs and updating the LCD.
\begin{lstlisting}[language=C]
int main(void) {
lcd_init();
lcd_clear();
lcd_print("Calculator Ready");
_delay_ms(1000);
lcd_clear();

while (1) {
// Read and process button inputs
updateLCD();
_delay_ms(10);
}
return 0;
}
\end{lstlisting}

Initializes the LCD and enters a loop to handle user inputs.
\newpage