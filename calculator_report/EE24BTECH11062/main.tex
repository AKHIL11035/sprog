\documentclass[journal]{IEEEtran}
\usepackage[a5paper, margin=10mm, onecolumn]{geometry}
\usepackage{cite}
\usepackage{amsmath,amssymb,amsfonts,amsthm}
\usepackage{algorithmic}
\usepackage{graphicx}
\usepackage{textcomp}
\usepackage{xcolor}
\usepackage{listings}
\usepackage{enumitem}
\usepackage{hyperref}
\usepackage{mathtools}
\usepackage{gensymb}

\begin{document}

\title{
Hardware Assignment \\ Scientific Calculator

\large{EE1003}
}

\author{Homa Harshitha Vuddanti \\(EE24BTECH11062)}

\maketitle

\bigskip

\textbf{Problem Statement}: Design and implement a scientific calculator using AVR-GCC and an LCD display, capable of performing basic arithmetic operations and scientific functions (preferably using differential equations) without relying on external computational libraries.

\section{Introduction}
This project focuses on developing a scientific calculator using the ATmega328P microcontroller, interfaced with a 16x2 LCD display and input push-buttons. The calculator supports basic arithmetic operations (addition, subtraction, multiplication, division) and scientific computations such as sine, cosine, logarithm, square root, and arctangent calculations using numerical methods like Runge-Kutta approximation. The system also includes a mode-switching feature between standard arithmetic and scientific functions.

\section{Materials Required}
\begin{itemize}
    \item ATmega328P (Arduino Uno Board)
    \item LCD Display
    \item Breadboard
    \item Potentiometer (for controlling contrast)
    \item Resistors (for button input stabilization)
    \item Push Buttons (for numeric, operator, and mode control inputs)
    \item Connecting Wires
\end{itemize}

\section{Procedure}
\subsection{Hardware Setup}
\begin{itemize}
    \item The 16x2 LCD display is connected to PORTD (pins PD2 to PD7) of the ATmega328P for RS, EN, and data lines.
    \item Push buttons are connected to PORTB and PORTC pins, each designated for numbers (0--9), arithmetic operators (+, -, *, /), and scientific functions (sin, cos, ln, sqrt, atan).
    \item Pull-up resistors are configured to ensure stable button readings.
    \item The mode switch button toggles between normal and scientific calculation modes.
\begin{table}[h!]
\centering
\caption{Hardware Connections}
\begin{tabular}{|c|c|c|}
\hline
\textbf{Component} & \textbf{Pin(s)} & \textbf{Description} \\
\hline
LCD RS & PD2 & Register Select \\
LCD EN & PD3 & Enable \\
LCD Data Lines (D4–D7) & PD4–PD7 & LCD 4-bit Data Interface \\
\hline
Button 0 & PB0 & Number 0 / Addition (+) in sci mode \\
Button 1 & PB1 & Number 1 / Subtraction (-) in sci mode \\
Button 2 & PB2 & Number 2 / Multiplication (×) in sci mode \\
Button 3 & PB3 & Number 3 / Division (÷) in sci mode \\
Button 4 & PB4 & Number 4 / $\sin(x)$ in sci mode \\
Button 5 & PB5 & Number 5 / $\cos(x)$ in sci mode \\
Button 6 & PC0 & Number 6 / $\ln(x)$ in sci mode \\
Button 7 & PC1 & Number 7 / $\sqrt{x}$ in sci mode \\
Button 8 & PC2 & Number 8 / $\arctan(x)$ in sci mode \\
Button 9 & PC3 & Number 9 \\
Equal button & PC4 & Compute Result \\
Mode toggle button (2nd) & PC5 & Switch between Normal and Scientific mode \\
\hline
\end{tabular}
\end{table}

\end{itemize}

\subsection{Software Implementation}
The software is structured with modular functions and interrupt-driven button handling. Key components include:
\begin{itemize}
    \item \textbf{LCD Driver Functions}: Functions for initializing the LCD, sending commands, displaying characters, and printing strings.
    \item \textbf{Numerical Methods}: Implementation of Runge-Kutta 4th order methods for calculating sine, cosine, logarithm, square root, and arctangent values with high precision for smaller values.
    \item \textbf{Button Reading and Debouncing}: Detection of button presses with debounce logic to prevent false triggers.
    \item \textbf{Main Loop}: Manages input state, detects operator entry, accumulates numbers, switches modes, and triggers calculations upon confirmation.
    \item \textbf{Output Display}: Results are displayed on the LCD with four-decimal-place precision for scientific computations.
\end{itemize}

\section{Button Description}
\begin{enumerate}
    \item Buttons mapped to PB0--PB5 and PC0--PC5 handle number inputs, operator selection, and scientific functions.
    \item A dedicated button toggles between standard and scientific modes.
    \item In scientific mode, number keys correspond to functions: sin(x), cos(x), ln(x), sqrt(x), and atan(x).
    \item The equal button (PC4) executes the computation and displays the result.
    \item so basically, 
    \begin{table}[h!]
\centering
\caption{Button Mapping in Normal and Scientific Modes}
\begin{tabular}{|c|c|c|}
\hline
\textbf{Button (Pin)} & \textbf{Normal Mode Function} & \textbf{Scientific Mode Function} \\
\hline
PB0 & 0 & Addition (+) \\
PB1 & 1 & Subtraction (-) \\
PB2 & 2 & Multiplication (×) \\
PB3 & 3 & Division (÷) \\
PB4 & 4 & $\sin(x)$ \\
PB5 & 5 & $\cos(x)$ \\
PC0 & 6 & $\ln(x)$ \\
PC1 & 7 & $\sqrt{x}$ \\
PC2 & 8 & $\arctan(x)$ \\
PC3 & 9 & none \\
PC4 & Equal (=) & Equal (=) \\
PC5 & Mode toggle (2nd) & Mode toggle (2nd) \\
\hline
\end{tabular}
\end{table}
\section{Specific features and usage:}
\item The lcd displays "Ready!" followed by the mode it is in when we turn it on by uploading the code.
\item When we press on the toggle button, the mode gets changed and that change is shown on lcd constantly throughout.
\item The calculator has 0-9 buttons when in normal mode, when we press on toggle button, it's button mappings change as the table above. This has been done because there aren't enough Arduino pins to allot for every button.
\item We can press our required operation and it is coded such that after selecting an operation, it automatically switches back to normal mode so that we can press our second number without repeatedly pressing on toggle button.
\item We can view the output when we press on equal button.
\item The functions implemented are : $\sin$, $\cos$, $\ln$, square root, $\tan$ inverse, using RK4 method for solving differential equations.
\end{enumerate}
\section{Improvements}
The $\ln$, $sqrt$, $\tan$ inverse functions are accurate for smaller values but not for larger values. It can be made better by improving the RK4 method.
\section{Code}
The complete code is available in the following GitHub permalink: \\
\url{https://github.com/HomaHarshitha/Scientific-calculator/tree/0bf909d96f977807a1d9feecd3566ab3c2201ca3/Scientific-calculator-ee062}

\section{Result}
The scientific calculator successfully performs both standard arithmetic and advanced scientific functions using user input provided through push-buttons. The results are clearly displayed on a 16x2 LCD screen with an intuitive interface for mode switching and calculation. The implemented numerical methods ensure accurate approximations and the calculator operates reliably in real time with efficient resource usage.

\end{document}


