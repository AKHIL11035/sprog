\documentclass[article]{IEEEtran}
\usepackage[a5paper, margin=10mm, onecolumn]{geometry}
\usepackage{tfrupee}

\setlength{\headheight}{1cm}
\setlength{\headsep}{0mm}

\usepackage{gvv-book}
\usepackage{gvv}
\usepackage{cite}
\usepackage{amsmath,amssymb,amsfonts,amsthm}
\usepackage{algorithmic}
\usepackage{graphicx}
\usepackage{textcomp}
\usepackage{xcolor}
\usepackage{txfonts}
\usepackage{listings}
\usepackage{enumitem}
\usepackage{mathtools}
\usepackage{gensymb}
\usepackage{comment}
\usepackage[breaklinks=true]{hyperref}
\usepackage{tkz-euclide} 
\usepackage{listings}                                       
\def\inputGnumericTable{}                                 
\usepackage[latin1]{inputenc}                                
\usepackage{color}                                            
\usepackage{array}                                            
\usepackage{longtable}                                       
\usepackage{calc}                                             
\usepackage{multirow}                                         
\usepackage{hhline}                                           
\usepackage{ifthen}                                           
\usepackage{lscape}

\renewcommand{\thefigure}{\theenumi}
\renewcommand{\thetable}{\theenumi}
\setlength{\intextsep}{10pt}

\numberwithin{figure}{enumi}
\renewcommand{\thetable}{\theenumi}

\begin{document}
\bibliographystyle{IEEEtran}
\title{10.4.4.5}
\author{EE24BTECH11035 - KOTHAPALLI AKHIL}
{\let\newpage\relax\maketitle}

\noindent\textbf{Question:}  
Is it possible to design a rectangular park of perimeter $80 \, \text{m}$ and area $400 \, \text{m}^2$? If so, find its length and breadth. \\

\noindent\textbf{Solution:} \\
\textbf{Theoritical solution :}\\
Let the length and breadth of the rectangular park be $l$ and $b$, respectively. The perimeter of the rectangle is given by:
\begin{align}
    2(l + b) = 80.
\end{align}
Simplifying, we get:
\begin{align}
    l + b = 40. 
\end{align}

The area of the rectangle is given by:
\begin{align}
    l \cdot b = 400. 
\end{align}

From first Equation, we can express $b$ in terms of $l$:
\begin{align}
    b &= 40 - l 
\end{align}

Substituting into Area Equation, we get:
\begin{align}
    l(40 - l) = 400
\end{align}

Expanding and rearranging terms:
\begin{align}
    l^2 - 40l + 400 = 0
\end{align}

Equation  is a quadratic equation. Solving it using the quadratic formula:
\begin{align}
    l &= \frac{-(-40) \pm \sqrt{(-40)^2 - 4(1)(400)}}{2(1)} \\
    l &= \frac{40 \pm \sqrt{1600 - 1600}}{2} \\
    l &= \frac{40 \pm 0}{2} \\
    l &= 20.
\end{align}

Substituting $l = 20$ , we find:
\begin{align}
    b &= 40 - 20 \\
    b &= 20
\end{align}

Thus, the rectangular park is a square with side length $20 \, \text{m}$. \\
 
Yes, it is possible to design a rectangular park with the given dimensions. The length and breadth are both $20 \, \text{m}$.\\
\textbf{Computational approach: }
\begin{itemize}
\item Using Fixed point iteration method,\\
Given equation:  
\begin{align}
    l(40 - l) = 400
\end{align}

Rewriting the equation in standard form:  
\begin{align}
    f(l) = l(40 - l) - 400 = 0
\end{align}

To apply the fixed-point iteration method, we express $l$ as a function $g(l)$:  
\begin{align}
    l = g(l)
\end{align}

From the given equation, we can rewrite $l$ in an iterative form:  
\begin{align}
    l_{n+1} = \frac{400}{40 - l_n}
\end{align}

Here, $l_{n+1}$ is the updated value in terms of the previous value $l_n$.\\
Steps to Perform Fixed-Point Iteration,\\
1. Choose an initial guess $l_0$.For, this problem the initial guess i took is 10\\
2. Compute successive iterations using the formula:  
   \begin{align}
       l_{n+1} = \frac{400}{40 - l_n}
   \end{align}
3. Check if the sequence $\{l_n\}$ converges to a fixed value. If it converges, the fixed point is a root of the equation.\\
Convergence Check,\\
For the fixed-point iteration to converge, the derivative of $g(l)$ must satisfy:  
\begin{align}
    \left| g'(l) \right| < 1 \quad \text{in the interval of interest}.
\end{align}

Differentiating $g(l)$:  
\begin{align}
    g(l) = \frac{400}{40 - l}, \quad g'(l) = \frac{400}{(40 - l)^2}
\end{align}

We verify the convergence condition by ensuring $\left| g'(l) \right| < 1$ near the expected root.

Conclusion:\\
If the iterations $\{l_n\}$ converge, the fixed point represents a root of the equation $l(40 - l) = 400$. If not, no root exists or the method fails to converge.\\
The output when I  executed the code of above approach,\\
\texttt{The approximate root is: 19.980040}

\newpage

\item Using QR decomposition of companion matrix to find the Roots of the quadratic equation,\\
\\To solve the equation $l(40-l) = 400$ using the QR decomposition method, we transform the problem into finding the eigenvalues of a companion matrix. The eigenvalues of the matrix correspond to the roots of the quadratic equation.

Step 1: Rearrange the Equation\\
The equation $l(40-l) = 400$ can be rewritten as:
\begin{align}
l^2 - 40l + 400 &= 0.
\end{align}

This is a quadratic equation where $a = 1$, $b = -40$, and $c = 400$.
Step 2: Construct the Companion Matrix\\
For the quadratic equation $ax^2 + bx + c = 0$, the companion matrix is given by:
\begin{align}
A &= 
\begin{bmatrix}
0 & 1 \\
-\frac{c}{a} & -\frac{b}{a}
\end{bmatrix}.
\end{align}

Substituting $a = 1$, $b = -40$, and $c = 400$, we get:
\begin{align}
A &= 
\begin{bmatrix}
0 & 1 \\
-\frac{400}{1} & -\frac{-40}{1}
\end{bmatrix} =
\begin{bmatrix}
0 & 1 \\
-400 & 40
\end{bmatrix}.
\end{align}

Step 3: QR Decomposition\\
The QR decomposition involves factorizing $A$ into:
\begin{align}
A_n &= Q_n R_n,
\end{align}
where $Q_n$ is an orthogonal matrix and $R_n$ is an upper triangular matrix.

Step 4: Iterative Update of the Matrix\\
At each iteration, update $A$ as:
\begin{align}
A_{n+1} &= R_n Q_n.
\end{align}

Repeat this process until $A_n$ converges to an upper triangular matrix. The diagonal elements of the final $A_n$ are the eigenvalues, which are the roots of the quadratic equation.

Step 5: Roots of the Equation\\
After sufficient iterations, the eigenvalues of $A$ converge to the roots of the quadratic equation $l^2 - 40l + 400 = 0$. The roots are:
\begin{align}
l &= 20 \quad \text{(repeated root)}.
\end{align}
on running the code for above QR approach the output is ,\\
\texttt{Roots of the quadratic equation: [20.20000494 19.79999506]}
\end{itemize}
\end{document}

