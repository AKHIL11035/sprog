\documentclass[journal]{IEEEtran}
\usepackage[a5paper, margin=10mm, onecolumn]{geometry}
%\usepackage{lmodern} % Ensure lmodern is loaded for pdflatex
\usepackage{tfrupee} % Include tfrupee package

\setlength{\headheight}{1cm} % Set the height of the header box
\setlength{\headsep}{0mm}     % Set the distance between the header box and the top of the text

\usepackage{gvv-book}
\usepackage{gvv}
\usepackage{cite}
\usepackage{amsmath,amssymb,amsfonts,amsthm}
\usepackage{algorithmic}
\usepackage{graphicx}
\usepackage{textcomp}
\usepackage{xcolor}
\usepackage{txfonts}
\usepackage{listings}
\usepackage{enumitem}
\usepackage{mathtools}
\usepackage{gensymb}
\usepackage{comment}
\usepackage[breaklinks=true]{hyperref}
\usepackage{tkz-euclide} 
\usepackage{listings}
% \usepackage{gvv}                                        
\def\inputGnumericTable{}                                 
\usepackage[latin1]{inputenc}                                
\usepackage{color}                                            
\usepackage{array}                                            
\usepackage{longtable}                                       
\usepackage{calc}                                             
\usepackage{multirow}                                         
\usepackage{hhline}                                           
\usepackage{ifthen}                                           
\usepackage{lscape}
\begin{document}

\bibliographystyle{IEEEtran}
\vspace{3cm}

\title{10.4.1.1.8}
\author{EE24BTECH11024 - G. Abhimanyu Koushik}
 \maketitle
% \newpage
% \bigskip
{\let\newpage\relax\maketitle}

\renewcommand{\thefigure}{\theenumi}
\renewcommand{\thetable}{\theenumi}
\setlength{\intextsep}{10pt} % Space between text and floats


\numberwithin{equation}{enumi}
\numberwithin{figure}{enumi}
\renewcommand{\thetable}{\theenumi}


\textbf{Question}:\\
Find the roots of the equation $x^3 - 4x^2 - x + 1 = \brak{x - 2}^3$
\\
\textbf{Solution: }\\
Theoritical solution:\\
The equation can be simplified to
\begin{align}
	x^3 - 4x^2 - x + 1 &= \brak{x - 2}^3\\
	x^3 - 4x^2 - x + 1 &= x^3-6x^2+12x-8\\
	2x^2 - 13x + 9 &= 0
\end{align}
Applying quadratic formula gives solution as
\begin{align}
	x_1 = \frac{13-\sqrt{97}}{4}\\
	x_2 = \frac{13+\sqrt{97}}{4}
\end{align}
Computational solution:\\
Two methods to find solution of a quadratic equation are:\\
Matrix-Based Method:\\
For a polynomial equation of form $x_n+b_{n-1}x^{n-1}+\dots+b_2x^2+b_1x+b_0 = 0$ we construct a matrix called companion matrix of form
\begin{align}
	\Lambda = \myvec{0&1&0&\dots&0\\ 0&0&1&\dots&0\\ \vdots &\vdots &\vdots &\ddots&\vdots\\0&0&0&\vdots&1\\-b_0&-b_1&-b_2&\dots&-b_{n-1}}
\end{align}
The eigenvalues of this matrix are the roots of the given polynomial equation.\\
The solution given by the code is
\begin{align}
	x_1 = 0.7878\\
	x_2 = 5.7122
\end{align}
Newton-Raphson Method:\\
Start with an initial guess $x_0$, and then run the following logical loop,
\begin{align}
    x_{n+1} = x_n - \frac{f\brak{x_n}}{f^{\prime}\brak{x_n}} 
\end{align}
where,
\begin{align}
    f\brak{x} = 2x^2 - 13x + 9\\
    f^{\prime}\brak{x} = 4x-13
\end{align}
The update equation will be
\begin{align}
	x_{n+1} = x_n - \frac{2{x_n}^2 - 13x_n + 9}{4x_n-13}\\
\end{align}
The problem with this method is if the roots are complex but the coeffcients are real, $x_n$ either converges to an extrema or grows continuously without any bound.
To get the complex solutions, however , we can just take the initial guess point to be a 
random complex number.\\
The output of a program written to find roots is shown below:
\begin{align}
	r_1 = 0.7878\\
	r_2 = 5.7122
\end{align}
Companion matrix
\begin{align}
\Lambda_{0} &= \myvec{0 & 1\\-\frac{9}{2} & \frac{13}{2}}
\end{align}
The update equation will be
\begin{align}
\Lambda_{n+1} &= Q\brak{\Lambda_{n}-\sigma I}Q^\top + \sigma I\\
\end{align}
As $n$ tends to infinite, $\Lambda$ converges to Upper triangular matrix
\begin{align}
\sigma &= \frac{13}{2}\\
\Lambda_{\text{shifted}} &= \myvec{0 & 1\\-\frac{9}{2} & \frac{13}{2}} - \sigma I\\
\Lambda_{\text{shifted}} &= \myvec{\frac{-13}{2} & 1\\ \frac{-9}{2} & 0}
\end{align}
Where $\sigma$ is the last diagonal element.
\begin{align}
Q &= \myvec{c & s\\-s & c}\\
\end{align}
Where
\begin{align}
c &= \cos{\phi}\\
s &= \sin{\phi}\\
	\myvec{c & s\\-s & c} \times \myvec{\frac{-13}{2} & 1\\ \frac{-9}{2} & 0} &= \myvec{a & b\\ 0 & c}\\
	c\brak{\frac{9}{2}} + s\brak{\frac{13}{2}} &= 0\\
	c^2+s^2 &= 1
\end{align}
Solving for $c$ and $s$ gives
\begin{align}
	c &= \frac{\frac{-13}{2}}{\sqrt{\brak{\frac{13}{2}}^2+\brak{\frac{9}{2}}}^2}\\
	s &= \frac{\frac{-9}{2}}{\sqrt{\brak{\frac{13}{2}}^2+\brak{\frac{9}{2}}}^2}
\end{align}
Now, as we got the $Q$ matrix we will do the following
\begin{align}
\Lambda_{\text{new}} &= Q\Lambda_{\text{shifted}}Q^\top + \sigma I\\
\Lambda_{\text{new}} &= \myvec{c & s\\-s & c}\myvec{\frac{-13}{2} & 1\\ \frac{-9}{2} & 0}\myvec{c & -s\\s & c} + \sigma I\\
\Lambda_{\text{new}} &\approx \myvec{0.468 & 5.176\\-0.324 & 6.032} 
\end{align}
Run the same sequence of steps for 20 iterations after which you end up with the following matrix
\begin{align}
    \Lambda_{\text{new}} = \myvec{0.78778555 & 5.5\\ 0 & 5.71221445}
\end{align}
The eigenvalues are same as its diagonal elements\\
Hence the roots of given equation are 0.78778555 and 5.71221445\newline \newline
To reduce a matrix into $QR$ form we do the following\newline
For a general matrix, we should bring to hessenberg form $H$, for that let
\begin{align}
	P = I-2\vec{w}\vec{w}^\top
\end{align}
where w is a vector with $\abs{w}^2=1$. The matrix $P$ is orthogonal as
\begin{align}
P^2 &=\brak{I - 2\vec{w}\vec{w}^\top}\cdot\brak{I - 2\vec{w}\vec{w}^\top}\\
    &=I-4\vec{w}\vec{w}^\top + 4\vec{w}\cdot\brak{\vec{w}^\top\vec{w}^\top}\cdot\vec{w}^\top\\
    &=I
\end{align}
Therefore, $P=P^{-1}$ but $P = P^\top$, so $P = P^\top$ \\
We can rewrite $P$ as
\begin{align}
	P = I - \frac{\vec{u}\vec{u}^\top}{H}
\end{align}
where the scalar $H$ is
\begin{align}
H = \frac{1}{2}\abs{\vec{u}}^2
\end{align}
Where $\vec{u}$ can be any vector. Suppose $\vec{x}$ is the vector composed of the first column of $A$. Take
\begin{align}
	\vec{u} = \vec{x} \mp \abs{\vec{x}}\vec{e}_1
\end{align}
Where $\vec{e}_1 = \myvec{1 & 0 & \dots}^\top$, we will take the choice of sign later. Then
\begin{align}
	P\cdot\vec{x} &= \vec{x} - \frac{\vec{u}}{H}\cdot\brak{\vec{u}\mp \abs{\vec{x}}\vec{e}_1}^\top\cdot \vec{x}\\
	              &= \vec{x} - \frac{2\vec{u}\brak{\abs{x}^2\mp \abs{x}x_1}}{2\abs{x}^2\mp \abs{x}x_1}\\
	              &= \vec{x} - \vec{u}\\
	              &= \mp\abs{\vec{x}}\vec{e}_1
\end{align}
To reduce a matrix $A$ into Hessenberg form, we choose vector $\vec{x}$ for the first householder matrix to be lower $n-1$ elements of the first column, then the lower $n-2$ elements will be zeroed.
\begin{align}
P_1\cdot A &= \myvec{
1 & 0 & 0 & \cdots & 0 \\
0 & p_{11} & p_{12} & \cdots & p_{1n} \\
0      & p_{21} & p_{22} & \cdots & p_{2n} \\
0      & p_{31} & p_{32} & \cdots & p_{3n} \\
\vdots & \vdots & \vdots & \ddots & \vdots \\
0      & p_{n1} & p_{n2} & \cdots & p_{nn}
}\myvec{
a_{00} & \times & \times & \cdots & \times \\
a_{10} & \times & \times & \cdots & \times \\
a_{20} & \times & \times & \cdots & \times \\
a_{30} & \times & \times & \cdots & \times \\
\vdots & \vdots & \vdots & \ddots & \vdots \\
a_{n0} & \times & \times & \cdots & \times
}\\
&=\myvec{
a_{00}^\prime & \times & \times & \cdots & \times \\
0 & \times & \times & \cdots & \times \\
0      & \times & \times & \cdots & \times \\
0      & \times   & \times & \cdots & \times \\
\vdots & \vdots & \vdots & \ddots & \vdots \\
0      & \times      & \times      & \cdots & \times
}
\end{align}
Now we choose the vector $\vec{x}$ for the householder matrix to be the bottom $n-2$ elements of the second column, and from it construct the $P_2$
\begin{align}
P_2 = \myvec{
1 & 0 & 0 & \cdots & 0 \\
0 & 1 & 0 & \cdots & 0 \\
0 & 0 & p_{22} & \cdots & p_{2n} \\
0 & 0 & p_{32} & \cdots & p_{3n} \\
\vdots & \vdots & \vdots & \ddots & \vdots \\
0 & 0 & p_{n2} & \cdots & p_{nn}
}
\end{align}
Continue the pattern\\
Let $H$ be the Hessenberg matrix. To find $Q$, we do the following, let
\begin{align}
G = \myvec{
1 & \cdots & 0 & 0 & \cdots & 0 \\
\vdots & \ddots & \vdots & \vdots & \reflectbox{$\ddots$} & \vdots \\
0 & \cdots & c & s & \cdots & 0\\
0 & \cdots & -s & c & \cdots & 0\\
\vdots & \reflectbox{$\ddots$} & \vdots & \vdots & \ddots & \vdots\\
0 & \cdots & 0 & 0 & \cdots & 1
}
\end{align}
Where the value of $c$ and $s$ are
\begin{align}
	c = \frac{\overline{x_{i,i}}}{\sqrt{\abs{x_{i,i}}^2+\abs{x_{i,i+1}}^2}}\\
	s = \frac{\overline{x_{i,i+1}}}{\sqrt{\abs{x_{i,i}}^2+\abs{x_{i,i+1}}^2}}
\end{align}
Multiplying $G$ and $A$ nulls out the element in $\brak{i+1}^{\text{th}}$ row and $i^\text{th}$ column. Now
\begin{align}
	Q = G_{n-1}G_{n-2}\cdots G_2G_1
\end{align}
\newline
\textbf{QR decomposition:}\newline
Given $H_0$
\begin{align}
	H_n = Q_nR_n\\
	Q^\top H_n = R_n\\
	H_{n+1} = R_nQ_n
\end{align}
The update equation will be
\begin{align}
	H_{n+1} = Q^\top H_n Q
\end{align}
As $n$ tends to infinite, $H$ will converge to upper triangular matrix, whose eigenvalues are the roots of the equation\newline
\textbf{QR decomposition with Shift:}\newline
Given $H_0$
\begin{align}
	H_n - \sigma I = Q_nR_n\\
	Q^\top\brak{H_n - \sigma I} = R_n\\
	H_{n+1} = R_nQ_n + \sigma I
\end{align}
The update equation will be
\begin{align}
H_{n+1} = Q^\top\brak{H_n - \sigma I}Q_n + \sigma I
\end{align}
Where $\sigma$ is the last diagonal element
As $n$ tends to infinite, $H$ will converge to upper triangular matrix faster than QR decomposition
\end{document}
