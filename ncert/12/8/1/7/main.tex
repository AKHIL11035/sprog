\let\negmedspace\undefined
\let\negthickspace\undefined
\documentclass[journal]{IEEEtran}
\usepackage[a5paper, margin=10mm, onecolumn]{geometry}
%\usepackage{lmodern} % Ensure lmodern is loaded for pdflatex
\usepackage{tfrupee} % Include tfrupee package

\setlength{\headheight}{1cm} % Set the height of the header box
\setlength{\headsep}{0mm}     % Set the distance between the header box and the top of the text

\usepackage{gvv-book}
\usepackage{gvv}
\usepackage{cite}
\usepackage{amsmath,amssymb,amsfonts,amsthm}
\usepackage{algorithmic}
\usepackage{graphicx}
\usepackage{textcomp}
\usepackage{xcolor}
\usepackage{txfonts}
\usepackage{listings}
\usepackage{enumitem}
\usepackage{mathtools}
\usepackage{gensymb}
\usepackage{comment}
\usepackage[breaklinks=true]{hyperref}
\usepackage{tkz-euclide} 
\usepackage{listings}
% \usepackage{gvv}                                        
\def\inputGnumericTable{}                                 
\usepackage[latin1]{inputenc}                                
\usepackage{color}                                            
\usepackage{array}                                            
\usepackage{longtable}                                       
\usepackage{calc}                                             
\usepackage{multirow}                                         
\usepackage{hhline}                                           
\usepackage{ifthen}                                           
\usepackage{lscape}

\usepackage{multicol}

% Marks the beginning of the document
\begin{document}
\bibliographystyle{IEEEtran}
\vspace{3cm}

\title{12.8.1.7}
\author{EE24BTECH11018 - Durgi Swaraj Sharma}

% \maketitle
% \newpage
% \bigskip
{\let\newpage\relax\maketitle}
\renewcommand{\thefigure}{\theenumi}
\renewcommand{\thetable}{\theenumi}
\setlength{\intextsep}{10pt}
\numberwithin{equation}{enumi}
\numberwithin{figure}{enumi}
\renewcommand{\thetable}{\theenumi}

\textbf{Exercise 8.1.7} Find the area of the smaller part of the circle $x^2+y^2=a^2$ cut off by the line $x=\frac{a}{\sqrt{2}}$.\\
\textbf{Theroretical solution}
\begin{align}
	x^2+y^2=a^2\\
	y^2=a^2-x^2
\end{align}
Taking square root on both sides,
\begin{align}
	y = \pm\sqrt{a^2-x^2}\\
\end{align}
Exploiting symmetry of our problem along the $x$-axis,
\begin{align}
	y = \sqrt{a^2-x^2} \label{eq1}
\end{align}
The smaller part of the circle cut by $x=\frac{a}{\sqrt{2}}$ is the region between $x=\frac{a}{\sqrt{2}} and x = a$. So we find the area of this region as follows.
\begin{align}
	\text{New area}\, &= \cdot\int\limits_{\frac{a}{\sqrt{2}}}^{a} \sqrt{a^2-x^2}
\end{align}
Changing to polar coordinates,
\begin{align}
	x = a \cos\theta\\
	dx = - a \sin\theta d\theta\\
	\text{Plugging into our integral,}\, &\int\limits_{\frac{\pi}{4}}^{0} \sqrt{a^2-a^2 \cos^2\theta} \brak{-a \sin\theta} d\theta\\
	&= \int\limits_{\frac{\pi}{4}}^{0} \sqrt{a^2 \sin^2\theta} \brak{-a \sin\theta} d\theta\\
	&= \int\limits_{\frac{\pi}{4}}^{0} a \sin\theta \brak{-a \sin\theta} d\theta
\end{align}
\begin{align}
	&= -\int\limits_{\frac{\pi}{4}}^{0} a^2 \sin^2\theta d\theta\\
	&= -a^2\sbrak{\frac{x}{2}-\frac{\sin2x}{4}}_{\frac{\pi}{4}}^{0}\\
	&= -a^2\sbrak{0 -\frac{\pi}{8} - 0 + \frac{1}{4}}\\
	\text{Required area}\, &= 2\cdot\text{New area} \, = 2\cdot a^2\cdot\frac{\pi-2}{8} = a^2\cdot\frac{\pi-2}{4}
\end{align}
We have theoretically found the area of the smaller part of the circle cut by the line to be $a^2 \frac{\pi-2}{4}$.\\
\textbf{Computational Solution}\\
To find the desired area computationally, we'll be utilising the Trapezoidal Rule.\\
Following from \ref{eq1}
\begin{align}
	y &= \sqrt{a^2-x^2} \nonumber\\
	\int\limits_{x_n}^{x_{n+1}} y \, dx &= \int\limits_{x_n}^{x_{n+1}} \sqrt{a^2-x^2} dx
\end{align}
We can solve the integral on the R.H.S. using the Trapezoidal Rule as follows.
\begin{align}
	A_{n+1}-A_{n} = 	\int\limits_{x_n}^{x_{n+1}} y \, dx &= h\sbrak{\frac{\sqrt{a^2-x_{n+1}^2} + \sqrt{a^2-x_n^2}}{2}}
\end{align}
Where $n$ is the number of iterations we want to calculate in, $h = \frac{a-\frac{a}{\sqrt{2}}}{n}$, $A_{n}$ is the area calculated till the $n^{th}$ iteration, and $x_0 = \frac{a}{\sqrt{2}}$.
The update equation for our area will be:
\begin{align}
	A_{n+1} = A_{n} +  h\sbrak{\frac{\sqrt{a^2-x_{n+1}^2} + \sqrt{a^2-x_n^2}}{2}}
\end{align}
The smaller our step-size, $h$, is, the more accurate our area calulation will be.\\
And the required area is twice of the calculated area, as the calculated region reflects about the $x$-axis in the total region.
\begin{align*}
	\text{Area calulcated when}\, a=13 \,\text{from:}\\
	\text{Theoretical Solution is}\, 48.232289\\
	\text{Computational Solution is}\, 48.232288
\end{align*}
\end{document}
