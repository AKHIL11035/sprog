\let\negmedspace\undefined
\let\negthickspace\undefined
\documentclass[journal]{IEEEtran}
\usepackage[a5paper, margin=10mm, onecolumn]{geometry}
%\usepackage{lmodern} % Ensure lmodern is loaded for pdflatex
\usepackage{tfrupee} % Include tfrupee package

\setlength{\headheight}{1cm} % Set the height of the header box
\setlength{\headsep}{0mm}     % Set the distance between the header box and the top of the text

\usepackage{gvv-book}
\usepackage{gvv}
\usepackage{cite}
\usepackage{amsmath,amssymb,amsfonts,amsthm}
\usepackage{algorithmic}
\usepackage{graphicx}
\usepackage{textcomp}
\usepackage{xcolor}
\usepackage{txfonts}
\usepackage{listings}
\usepackage{enumitem}
\usepackage{mathtools}
\usepackage{gensymb}
\usepackage{comment}
\usepackage[breaklinks=true]{hyperref}
\usepackage{tkz-euclide}
\usepackage{listings}                                     
\def\inputGnumericTable{}                                 
\usepackage[utf8]{inputenc}                                
\usepackage{color}                                            
\usepackage{array}                                            
\usepackage{longtable}                                       
\usepackage{calc}                                             
\usepackage{multirow}                                         
\usepackage{hhline}                                           
\usepackage{ifthen}                                           
\usepackage{lscape}
\renewcommand{\thefigure}{\theenumi}
\renewcommand{\thetable}{\theenumi}
\setlength{\intextsep}{10pt} % Space between text and floats

\numberwithin{equation}{enumi}
\numberwithin{figure}{enumi}
\renewcommand{\thetable}{\theenumi}

% Marks the beginning of the document
\begin{document}
\bibliographystyle{IEEEtran}

\title{Question-8.1.13}
\author{EE24BTECH11048-NITHIN.K} 
%\maketitle
%\newpage
%\bigskip
{\let\newpage\relax\maketitle}
\textbf{Question:} \\
Area of the region bounded by the curve $y^2 = 4x$, y-axis and the line $y = 3$ is \\

\textbf{Theoretical Solution:} \\
The Area bounded by the parabola, the line $y = 3$ and the y-axis is given by
\begin{align}
	&= \int_{0}^{3} \frac{y^2}{4}dy
\end{align}
\begin{align}
	&= \sbrak{\frac{y^3}{12}}_{0}^{3}
\end{align}
\begin{align}
	&= \frac{27}{12} = \frac{9}{4} = 2.25
\end{align}

\textbf{Computational Solution:} \\
Trapezoid Rule: It is a numerical method used to approximate the value of a definite integral. It is based on approximating the region under the curve by a series of trapezoids and then calculating the area of these trapezoids. \\
We discretize the range of y-coordinates with uniform step size $h \to 0$, such that the points are $y_0, y_1, y_2, \dots ,y_n$ and $y_{n+1} = y_n + h$ \\
Let the sum of all trapezoidal areas upto $y_n$ be $A_n$ and $f\brak{y} = \frac{y^2}{4}$ then the difference equation can be formed as \\
\begin{align}
	A_n &= \frac{h}{2}\brak{f\brak{y_0} + f\brak{y_1}} + \frac{h}{2}\brak{f\brak{y_1} + f\brak{y_2}} + \dots + \frac{h}{2}\brak{f\brak{y_{n-1}} + f\brak{y_{n}}}
\end{align}
\begin{align}
	A_n &= \frac{h}{2}\brak{f\brak{y_0} + f\brak{y_1}} + \frac{h}{2}\brak{f\brak{y_1} + f\brak{y_2}} + \dots + \frac{h}{2}\brak{f\brak{y_{n - 1}} + f\brak{y_{n}}}
\end{align}
\begin{align}
	A_{n+1} &= A_n + \frac{h}{2}\brak{f\brak{y_{n+1}} + f\brak{y_n}}
\end{align}
\begin{align}
	A_{n+1} &= A_n + \frac{h}{2}\brak{\frac{y_{n+1}^2}{4} + \frac{y_n^2}{4}}
\end{align}
Theoretical Area = 2.25 square units\\
Computational Area = 2.2500011 square units

\end{document}
